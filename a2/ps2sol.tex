\documentclass{assignment-263}

\usepackage{color}
\usepackage{pythontex}

\usepackage{mathtools}
\anum{2}
\course{CSC263 Winter 2020}
\duedate{Thursday, February 27, 2020}
\filename{ps2sol.pdf, ps2sol.tex and triangle.py}

\newcommand{\bigOh}{\mathcal{O}}

\begin{document}

%% HEADER %%
\vskip10pt
\textbf{Vivekanand Gadhia | 1004165253, \hskip7pt   Piotr Szaran | 1004171424 \hskip60pt Thursday, February 27th, 2020} 
\newline
\hrule 
\vskip30pt
%\think

  \begin{enumerate}

%  
\item[1.] \textbf{[total: 6]}
  You are given an array $\texttt{A}$ that contains $n$ \textbf{distinct}
  numbers. The task is to compute another array $\texttt{B}$ such that for all
  $\texttt{j},$ we want $\texttt{B[j]}$ to be the number of elements in
  $\texttt{A}$ that appear after $\texttt{A[j]}$ and are strictly
  smaller than $\texttt{A[j]}.$

  As an example, if you are given the input array
  $\texttt{A = [8, 10, 4, 7, 3, 5, 9, 1]},$ then you want to output
  the array $\texttt{B = [5, 6, 2, 3, 1, 1, 1, 0]}.$ This is because,
  in $\texttt{A},$ there are 5 elements that appear after
  $\texttt{A[0] = 8}$ and are smaller than \texttt{8} (i.e. \texttt{4,
    7, 3, 5, 1}), there are 6 elements that appear after
  $\texttt{A[1] = 10}$ and are smaller than \texttt{10}
  (i.e. \texttt{4, 7, 3, 5, 9, 1}), etc.
  

  Design an algorithm to compute $\texttt{B}$ given $\texttt{A}$ as
  input. The \textbf{worst case} complexity of your algorithm
  \textbf{must be $\bigOh(n\log n)$}.

  This algorithm can be described in brief given what we have learned
  in class. 
  
  Write a clear description of your algorithm (what data structure is
  used and how the algorithm utilizes the data structure) and justify
  its correctness and runtime. As usual, please use learned algorithms
  and analysis results from lectures and tutorials without repeating
  them. \newline \hrule \vskip14pt
  
  The data structure we decided to use was a Balanced BST, but for every node in the tree we include the size of the subtree of that node.\\
  
  The way our algorithm work is by initially traversing the inputted array from back to front inserting every key into the tree while maintaining the Balanced BST properties. As we insert every key into the tree, we compare the size of the key being inserted with the root of the tree. \\
  \begin{itemize}
  \item If the new node is larger, then we know that the minimum size of the new node will be the size of the roots left subtree, but we still have to traverse down the right subtree adding one for every node that is smaller than the one being inserted. \\
  
  \item If the new node is smaller, the new nodes size starts at zero, and we traverse down the left subtree adding one to the size for every node that is smaller. \\
  \end{itemize}
  
  This algorithm will yield a runtime of \textbf{O(nlogn)}, as required.
    
 \clearpage
  \item[2.] \textbf{[total: 6]} Suppose that we want to generate a BST by
  inserting the keys 1 through $n$ into an initially empty BST (assume
  $n > 1$). Assume that the insert sequence is a random permutation of
  $[1,2,3,\dots, n]$ and each permutation is equally likely to
  occur. Answer the following questions.
  %
  \begin{enumerate}[(a)]

  \item What is the maximum possible height of the generated BST?
    Describe \textbf{four} different insert sequences that would
    result in a BST with the maximum height. \vskip10pt
    
    The maximum possible height of a generated BST is 'n'. 
    \begin{itemize}
    \item $[n, (n-1), (n-2), ... , 3, 2, 1]$
    \item $[1, (n), (n-1), (n-2), ... , 3, 2]$
    \item $[1,2,3,...,(n-2),(n-1), n]$
    \item $[n,1,2,3,...,(n-2),(n-1)]$
    \end{itemize} 
    \vskip10pt

  \item By picking a random permutation of $[1,2,3,\dots, n]$ as the
    insert sequence, what is the probability that the resulting BST
    has the maximum height? Show detailed steps of your calculation
    with clear justification. \vskip10pt
    
    In order to generate a BST with a maximum possible height, the root node must either be the smallest number or the largest number. By doing this one child of the root will be empty, as required to achieve a maximum height. This implies that that there are 2C1 ways to pic the first number in the sequence. \\
    
    After choosing either 1 or n to be the first number in the sequence, we are left with either [2,3, ... , (n-1), n] or [1, 2, 3, ... ,(n-1)]. However, the idea from above can be generalized to subtrees. So we pick one out of the next largest and smallest numbers. This also has 2C1 ways. \\
    
    Applying this idea until we are out of keys to insert give us: $(2C1)^{n-1}$ ways, but 2C1 = 2! = 2 $\implies 2^{n-1}$. \\
    
    Further, there are n! ways to permute n objects, \textbf{therefore the probability} is $\frac{2^{n-1}}{n!}$ for n $>$ 1
  \end{enumerate}
 
\clearpage
\item[3.] \textbf{[total: 12]} In this problem, you will design the data
  structure for implementing an ADT called
  \texttt{RESTAURANT-SET}. Below is the description of the ADT.
%%%%%%%%%%%%%%%%%%%%%%%%%%%%%%%%%%%%%%%%
  \begin{description}
  \item[Objects:] A collection of restaurants in the same city. Each
    \texttt{restaurant} has the following attributes:
    \begin{itemize}
    \item \texttt{restaurant.name}: a string which is the name of the
      restaurant. Each restaurant has a unique name, i.e., no two
      restaurants have the same name.
    \item \texttt{restaurant.cost}: a decimal value such as $22.99$
      that denotes the cost of a 1 person meal at the restaurant.  For
      simplicity, assume that each restaurant's cost is unique,
      i.e., no two restaurants have the same cost.
    \item \texttt{restaurant.rating}: a decimal value between $0.0$ and
      $5.0$, e.g., $0.012$, $3.1415926$. For simplicity, assume that
      each restaurant's rating value is unique, i.e., no two
      restaurants have the same rating value.
    \end{itemize}
  \item[Operations:] \mbox{}
    \begin{itemize}
      %
    \item \texttt{GET-RESTAURANT(R, name)}: returns the restaurant
      with name \texttt{name} if it exists in the
      \texttt{RESTAURANT-SET R}; returns \texttt{NIL} if the restaurant
      does not exist in \texttt{R}.
      %
    \item \texttt{ADD-RESTAURANT(R, restaurant)}: Add a restaurant
      \texttt{restaurant} to the \texttt{RESTAURANT-SET R}. You may
      assume that \texttt{restaurant} has a unique name, a unique
      cost and a unique rating.
    \item \texttt{AVG-COST(R, r1, r2),\ r1 $\le$ r2}: considers all
      restaurants that have a rating $\texttt{r}$ such that
      $\texttt{r1} \le \texttt{r} < \texttt{r2},$ and returns the
      average of the meal costs over all these restaurants.

    \end{itemize}
    %
  \item[Requirements:] Let $n$ be the size of $R$,
    \begin{itemize}
    \item \texttt{GET-RESTAURANT(R, name)} must have average-case runtime
      $\mathcal{O}(1)$.
    \item \texttt{ADD-RESTAURANT(R, restaurant)} must have
      \textbf{\color{red} average-case} runtime $\mathcal{O}(\log n)$.
    \item \texttt{AVG-COST(R, r1, r2)} must have worst-case runtime
      $\mathcal{O}(\log n)$.
    \end{itemize}
  \end{description}
%%%%%%%%%%%%%%%%%%%%%%%%%%%%%%%%%%%%%%%%
  Give a \emph{detailed} description of the design of your data
  structure by answering the following questions.

  \begin{enumerate}[(a)]
  \item \textbf{[2]} Which data structure(s) do you use in your design? What is the
    key of each data structure? What attributes does each node in your
    data structure(s) keep? \vskip10pt
    
    We use two data structures in our code are \textbf{Hash tables and AVL's}
     \begin{itemize}
        \item Hash tables are used so that we can get an average case O(1) run time for  GET-RESTAURANT by using a hash search which  was shown in class to have average csae O(1).
        \item Balanced BST are built based on the restaurant ratings. Our balanced BST's are augmented so each node keeps track of the name, cost, and subtree's total cost. We use balanced BST so that the ADD-RESTAURANT function can have a average-case runtime of O(logn) because we showed in class that inserting into a balanced BST has this average-case runtime.
    \end{itemize}
    
    \vskip10pt
  \item \textbf{[2]} Explain concisely in English how your \texttt{GET-RESTAURANT}
    operation works, and justify its runtime. \vskip10pt
    
    Reiterating what was said above, our GET-RESTAURANT operation takes the input given by the user and by using a hash function will instantly return the name if the restaurant exists, or NIL if it does not. We can guarantee this has an average-case runtime of O(1) because of what we learned in lecture about simple uniform hashing. \vskip10pt
   

  \item \textbf{[2]} Explain concisely in English how your \texttt{ADD-RESTAURANT}
    operation works, and justify its runtime. In particular, explain
    why all the attributes kept in each node can be maintained
    efficiently upon the addition of the new node. \vskip10pt
    
    The ADD-RESTAURANT operation works like regular Balanced BST, with the node values being the rating of each restaurant (since no two rating can be the same by assumption). Therefore this function has the runtime of adding into a balanced BST, which was shown in class to have average-case runtime of O(logn). \vskip10pt
    \clearpage
  \item \textbf{[6]} Explain how your \texttt{AVG-COST} operation works and
    \textbf{write the pseudocode} of this operation, then justify its
    runtime. \vskip10pt
    
\begin{python}
#Main Function
def AVG-COST(R, r1, r2):
    if r1 < R.root.rating and r2<R.root.rating:
        return AVG-COST(R.root.left, r1, r2)
    else if r1 > R.root.rating and r2 > R.root.rating:
        return AVG-COST(R.root.right, r1, r2)
    else:
        min = search_min(R.root.left, r1)
        max = search_max(R.root.right, r2)
        return (min[1] + max[1] + R.root.price)/(min[0]+max[0] + 1)
        
#Helper Functions
def search_min(R, r1):
    '''
  search_min returns a tuple (number of elements in subtree, 
  total price of subtree) greater than/equal r1
  '''
  if R.root.rating > r1:
    tuple = search_min(R.root.left, r1)
    return (tuple[0]+(R.root.right.size)+1, tuple[1]+R.root.right.total+R.root.price)
  else if R.root.left.rating < r1:
    return search_min(R.root.right, r1)
  else:
    return ((R.root.right.size)+1, R.root.right.total+R.root.price)
    
def search_max(R, r2):
    '''
  search_max returns a tuple (number of elements in subtree, 
  total price of subtree) less than r2
  '''
  if R.root.rating > r2:
    return search_max(R.root.left, r2)
  else if R.root.left.rating < r2:
    tuple = search_max(R.root.right, r2)
    return (tuple[0]+(R.root.left.size)+1, tuple[1]+R.root.left.total+R.root.price)
  else:
    return ((R.root.left.size)+1, R.root.left.total+R.root.price)
\end{python}
    \vskip10pt
    \begin{itemize}
    \item The function AVG-COST is a modified version of search on an AVL tree and has been proven to have a runtime of O(log n). 
    \item The functions search\_ min and search\_ max are proven in class to have a worst-case runtime of O(log n) respectively.
    \item Therefore the total runtime of the function AVG-COST can be determined by adding the runtimes of the separate calls. The calls made on line $4, 6, 8, 9$ all have a runtime of O(log n). 
    \item$O(log n) + O(log n)+ O(log n) + O(log n) \implies 4O(log n) \imples$ a worst case runtime of O(log n) as required.
    \end{itemize}
  \end{enumerate}


  
 \clearpage
\program

\item[4.] \textbf{[total: 12]}
  In this problem, we will deal with the notion
  of \textbf{pseudo-similar} triangles.

  Each triangle is represented by a 3-tuple of positive numbers,
  specifying the sides of the triangle.

  We say that two triangles \texttt{t1} and \texttt{t2} are
  \textbf{pseudo-similar} if triangle \texttt{t2} can be obtained by
  rotating and reflecting triangle \texttt{t1}.

  For example, $\texttt{t1 = (6, 9, 12)}$ is \textbf{pseudo-similar}
  to $\texttt{t2 = (6, 12, 9)}$ since \texttt{t2} can be obtained by
  rotating and reflecting \texttt{t1}.
%
  \[ \texttt{(6, 9, 12)}
    \xrightarrow{\text{rotate}} \texttt{(9, 12, 6)}
    \xrightarrow{\text{reflect}} \texttt{(6, 12, 9)}
  \]

  However, $\texttt{t1 = (6, 9, 6)}$ is not \textbf{pseudo-similar} to
  $\texttt{t2 = (6, 9, 9)}.$

  We say that two triangles are the same {\bf kind} if they are
  \textbf{pseudo-similar}.

  In Python, a triangle will be represented as a 3-tuple of positive
  integers.

  Your task is to write the function \verb|num_triangle_kinds|, which
  determines the \textbf{number} of different kinds of triangles in
  the list. \textbf{(8 points)}


Requirements:
\begin{itemize}
\item Your code must be written in Python 3, and the filename must be \verb|triangle.py|.
\item We will grade only the \verb|num_triangle_kinds| function; please do not change its signature in the starter code. include as many helper functions as you wish.
   \item You are {\bf not} allowed to use the built-in Python
			dictionary or set.
			\item To get full marks, your algorithm must have average-case runtime
				$\mathcal{O}(n)$. You can assume Simple Uniform Random Hashing.
   \end{itemize}

\textbf{Write-up (4 points)}: in your \verb|ps3.pdf/ps3.tex| files, include the following: an explanation of how your code works, justification of correctness, and
 justification of desired $\mathcal{O}(n)$ average-case runtime.
  \newline \hrule \vskip10pt
  
  For our code we used simple uniform hashing, the multiplication method to be specific. \\\\
  For our constants, we let our A = $\frac{\sqrt{5}-1}{2}$ (as suggested by Donald Knuth), and $m=2^{32}$. However, since $2^{32} \approx$ 4.3 billion, and the multiplication method hashes a number between 1 and (m-1), the possibility of two numbers being hashed to the same number is so small that we can assume our average case operations are all O(1). \\
  
  Ultimately our hash equation looked like: \indent h(k) = $m*(kA(\text{mod }1))$, and we let k be  each triangle side, and then add and return h(side 1) + h(side 2) + h(side 3). This is so triangles (1,2,3) and (2,2,2) do not have the same hash, but triangles (3,6,12) and (12,3,6) do have the same hash. \\
  
  How our code works, is we begin by looping through every triangle and hashing each triangle. If the hashed triangle doesn't already exist in our list,  we add it, otherwise we are just looking at a pseudo-similar triangle and there is no need to add it to the list. The loop runs until every triangle is read and once the loop is done we return the length of the list.\\
  
  Since the operations in the loop take O(1) average-case time, and our loop runs 'n' times \textbf{we get a runtime of O(n)}.

 \end{enumerate}

\end{document}

%%% Local Variables:
%%% mode: latex
%%% TeX-master: t
%%% End:
